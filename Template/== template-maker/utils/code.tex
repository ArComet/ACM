\section{Basic}
\subsection{.vimrc}
\begin{lstlisting}[language=C++]
// vim ~/.vimrc
set nu ai ci si mouse=a ts=2 sts=2 sw=2
nmap<F3> : !gedit % <CR>
nmap<F8> : !time ./%< < %<.in <CR>
nmap<F5> :call CR()<CR>
func! CR()
exec "w"
exec "!g++ % -o %<"
exec "! ./%<"
endfunc
map<F2> :call SetTitle()<CR>
func SetTitle()
let l = 0
let l = l + 1 | call setline(l,'#include <bits/stdc++.h>')
let l = l + 1 | call setline(l,'using namespace std;')
endfunc
\end{lstlisting}
\subsection{head}
\begin{lstlisting}[language=C++]
#include<bits/stdc++.h>
using namespace std;
typedef long long ll;
typedef double db;
typedef pair<int,int> pii;
typedef vector<int> vi;
#define dd(x) cout << #x << "=" << x << ","
#define de(x) cout << #x << "=" << x << endl
#define rep(i,a,b) for(int i=(a);i<(b);++i)
#define per(i,a,b) for(int i=(b-1);i>=a;--i)
#define all(x) (x).begin(),(x).end()
#define sz(x) (int)(x).size()
#define mp make_pair
#define pb push_back
#define fi first
#define se second
#define endl "\n"
#define rk(x) upper_bound(all(V) , x) - V.begin()
#define lowbit(x) x&(-x)
#define inf 0x3f3f3f3f
const int N = 101010;
const int M = 1e9+7;

int main(){
	ios::sync_with_stdio(false);
	cin.tie(0);


	return 0;
}
\end{lstlisting}
\subsection{stl}
\begin{lstlisting}[language=C++]
#include<bits/stdc++.h>
using namespace std;
int main(){
	int num[6]={1,2,4,7,15,34},x=7;
	sort(num,num+6);// 从小到大排序 
	lower_bound(num,num+6,x);// 第一个大等于x的指针 
	upper_bound(num,num+6,x);// 第一个大于x的指针 
	sort(num,num+6,greater<int>());// 从大到小排序
	lower_bound(num,num+6,x,greater<int>());// 第一个小等于x的指针 
	upper_bound(num,num+6,x,greater<int>());// 第一个小于x的指针 
	return 0;	
}
\end{lstlisting}

\section{DataStructure}
\subsection{ST}
\begin{lstlisting}[language=C++]
// [0,n)
struct ST{
	static const int N = 101010;
	int a[20][N], lg[N];
	void build(int *v, int n){
		rep(i, 2, n + 1) lg[i] = lg[i >> 1] + 1;
		rep(i, 0, n) a[0][i] = v[i];
		rep(i, 1, lg[n] + 1) rep(j, 0, n - (1 << i) + 1) {
			a[i][j] = max(a[i - 1][j], a[i - 1][j + (1 << i >> 1)]);
		}
	}
	int qry(int l, int r){
		if(l > r) swap(l, r);
		int i = lg[r - l + 1];
		return max(a[i][l] , a[i][r + 1 - (1 << i)]);
	}
};
\end{lstlisting}

\section{Geo}
\subsection{基础点、向量}
\begin{lstlisting}[language=C++]
struct P {
	int quad() const { return sign(y) > 0 || (sign(y) == 0 && sign(x) >= 0); }
	P rot90() { return P(-y, x); }
	P rot(db a) { return P(cos(a) * x - sin(a) * y, cos(a) * y + sin(a) * x); }
	P norm() { return *this / len(); }
};
db rad(P p1, P p2) { return atan2l(det(p1, p2), dot(p1, p2)); } // p1 与 p2 的夹角,有方向
bool cmp(const pii &a, const pii &b) { // 级角排序
	int o = a > pii(0, 0), t = b > pii(0, 0);
	if(o != t) return o < t;
	return det(a, b) > 0;
}
// 【点集中最近点对】
namespace NearestPoints { // sz(A) <= 1e5
	db solve(int l, int r, vector<P> &p) {
		if(l == r) return 1e100;
		int m = l + r >> 1;
		db Xm = p[m].x, lim = min(solve(l, m, p), solve(m + 1, r, p));
		inplace_merge(p.begin() + l, p.begin() + m + 1, p.begin() + r + 1, [&](P a, P b){return a.y < b.y;});
		vector<P> V;
		rep(i, l, r + 1) if(fabs(p[i].x - Xm) <= lim) V.pb(p[i]);
		rep(i, 0, sz(V)) rep(j, i + 1, sz(V)) {
			if(fabs(V[j].y - V[i].y) >= lim) break;
			T dis = (V[i] - V[j]).len();
			lim = min(lim, dis);
		}
		return lim;
	}
	db solve(vector<P> A) {
		sort(all(A), [&](P a, P b){return a.x < b.x;});
		return solve(0, sz(A) - 1, A);
	}
}
// 【最小圆覆盖】
C Mincir(P *p,int n){ 
	random_shuffle(p , p + n);
	P o = p[0];db r = 0;
	rep(i,1,n) {
		if(sgn(abs(o-p[i])-r) <= 0) continue;
		o = p[i] , r = 0;
		rep(j,0,i) {
			if(sgn(abs(o-p[j])-r) <= 0) continue;
			o = (p[i] + p[j]) / 2 , r = abs(o-p[j]);
			rep(k,0,j) {
				if(sgn(abs(o-p[k])-r) <= 0) continue;
				o = outC(p[i],p[j],p[k]) , r = abs(o-p[k]);
			}}}
	return C(o,r);
}
// 【费马点】
// sqrt((a ^ 2 + b ^ 2 + c ^ 2 + 4 * sqrt(3) * area) / 2)
// 如果有重点,大于 2 的直接用模拟退火法
P fermat(vector<P> p) {
  int n = sz(p); assert(n);
  if(n == 1) return p[0];
  if(n == 2) return (p[0] + p[1]) / 2;
  if(n == 3) {
    db a[3];
    rep(i, 0, 3) a[i] = (p[(i + 2) % 3] - p[(i + 1) % 3]).len();
    rep(i, 0, 3) {
      int j = (i + 1) % 3, k = (i + 2) % 3;
      if(sign(a[i] * a[i] - a[j] * a[j] - a[k] * a[k] - a[j] * a[k]) >= 0) return p[i];
    }
    if(det(p[0], p[1], p[2]) < 0) swap(p[1], p[2]);
    P q1 = (p[2] - p[0]).rot(pi / 3) + p[0]; 
    P q2 = (p[0] - p[1]).rot(pi / 3) + p[1];
    return isLL(L(q1, p[1]), L(q2, p[2]));
  }
  auto Rand = [&] () { return rand() % 10000 / 5000 * pi; };
  P ans(0, 0); rep(i, 0, n) ans = ans + p[i]; ans = ans / n;
  db len = 0; rep(i, 0, n) len += (ans - p[i]).len();
  db t = 10000; // modify
  while(t > eps) {
    db ang = Rand();
    P np(ans.x + t * sin(ang), ans.y + t * cos(ang));
    db k = 0; rep(i, 0, n) k += (np - p[i]).len();
    if(sign(len - k) > 0) ans = np, len = k;
    t *= 0.999;
  }
  return ans;
}
\end{lstlisting}
\subsection{线段、直线、曲线}
\begin{lstlisting}[language=C++]
// 【直线交点】
P isLL(L l1, L l2) {
	db s1 = det(l2.b - l2.a, l1.a - l2.a);
	db s2 = -det(l2.b - l2.a, l1.b - l2.a);
	return (l1.a * s2 + l1.b * s1) / (s1 + s2);
}
P isLL(L l, db a, db b, db c) { // ax + by + c = 0
	db u = a * l.a.x + b * l.a.y + c;
	db v = -(a * l.b.x + b * l.b.y + c);
	return (l.a * v + l.b * u) / (u + v);
}
P isLL(db a0, db b0, db c0, db a1, db b1, db c1) {
	db d = a0 * b1 - a1 * b0;
	return P(b0 * c1 - b1 * c0, a1 * c0 - a0 * c1) / d;
}
// 【线相交判定】
bool isSSr(const L &a, const L &b){
	db c1 = det(a.t - a.s, b.s - a.s), c2 = det(a.t - a.s, b.t - a.s);
	db c3 = det(b.t - b.s, a.s - b.s), c4 = det(b.t - b.s, a.t - b.s);
	return sign(c1) * sign(c2) < 0 && sign(c3) * sign(c4) < 0;
}
bool isSS(L a,L b){
	db c1 = det(a.t - a.s, b.s - a.s), c2 = det(a.t - a.s, b.t - a.s);
	db c3 = det(b.t - b.s, a.s - b.s), c4 = det(b.t - b.s, a.t - b.s);
	return sign(c1) * sign(c2) <= 0 && sign(c3) * sign(c4) <= 0 &&
		sign(max(a.s.x, a.t.x) - min(b.s.x, b.t.x)) >= 0 &&
		sign(max(b.s.x, b.t.x) - min(a.s.x, a.t.x)) >= 0 &&
		sign(max(a.s.y, a.t.y) - min(b.s.y, b.t.y)) >= 0 &&
		sign(max(b.s.y, b.t.y) - min(a.s.y, a.t.y)) >= 0;
}
bool isLS(P a1, P a2, P b1, P b2) { // 判断直线线段是否相交(端点也算)
	db c1 = det(a2 - a1, b1 - a1), c2 = det(a2 - a1, b2 - a1);
	return sign(c1) * sign(c2) <= 0;
}
// 【点到线距离】
db disToL(L l, P p) {
  return fabs(det(l.a, p, l.b) / (l.b - l.a).len());
}
db disToS(L l, P p) {
  return sign(dot(l.a, p, l.b)) * sign(dot(l.b, p, l.a)) == 1 ? disToL(l, p) : min((p - l.a).len(), (p - l.b).len());
}
// 【线到线距离】
db disSS(L a, L b){
	if(isSS(a, b)) return 0;
	return min(min(disToSeg(b, a.s), disToSeg(b, a.t)), min(disToSeg(a, b.s), disToSeg(a, b.t)));
}
\end{lstlisting}
\subsection{凸包}
\begin{lstlisting}[language=C++]
// 【求凸包】
vector<P> convexHull(vector<P> ps) {
  int n = sz(ps); if(n <= 1) return ps;
  sort(all(ps)); vector<P> qs;
  for(int i = 0; i < n; qs.pb(ps[i++])) {
    while(sz(qs) > 1 && sign(det(qs[sz(qs) - 2], qs.back(), ps[i])) <= 0) qs.pop_back();
  }
  for(int i = n - 2, t = sz(qs); i >= 0; qs.pb(ps[i--])) {
    while(sz(qs) > t && sign(det(qs[sz(qs) - 2], qs.back(), ps[i])) <= 0) qs.pop_back();
  }
  qs.pop_back(); return qs;
}
// 【凸包最远点对】
db diameter(vector<P> A) {
	int n = sz(A);
	if(n <= 1) return 0;
	int l = 0, r = 0;
	rep(i, 1, n) (A[i] < A[l]) && (l = i), (A[r] < A[i]) && (r = i);
	db res = (A[l]-A[r]).len();
	int i = l, j = r;
	do (++(det(A[(i + 1) % n]- A[i], A[(j + 1) % n] - A[j]) >= 0 ? j : i)) %= n,
		res = max(res, (A[i] - A[j]).len());
	while(i != l || j != r);
	return res;
}
// 【动态凸包】
// O(nlogn)
// 插入点,询问点在不在凸包内部(包括边界)
namespace DCH {
	map<int, P> h1, h2;
	bool ao(P a, P b, P c) {
		// 包括边界:小等于
		return (b.y - a.y) * 1ll * (c.x - b.x) <= (c.y - b.y) * 1ll * (b.x - a.x);
	}
	bool in(map<int, P> &h, P p) {
		if(!sz(h)) return 0;
		if(p.x < h.begin()->se.x || p.x > h.rbegin()->se.x) return 0;
		auto l = h.lower_bound(p.x);
		if(p.x == l->se.x) return p.y <= l->se.y;
		auto r = l--;
		return ao(l->se, p, r->se);
	}
	void ins(map<int, P> &h, P p) {
		if(in(h, p)) return ;
		h[p.x] = p;
		auto pos = h.find(p.x);
		while(1) {
			auto l = pos; if(l == h.begin()) break; --l;
			auto ll = l; if(ll == h.begin()) break; --ll;
			if(ao(ll->se, l->se, p)) h.erase(l); else break;
		}
		while(1) {
			auto r = pos; r++; if(r == h.end()) break;
			auto rr = r; rr++; if(rr == h.end()) break;
			if(ao(p, r->se, rr->se)) h.erase(r); else break;
		}
	}
	void ins(int x, int y) { ins(h1, P(x, y)); ins(h2, P(x, -y)); }
	bool in(int x, int y) { return in(h1, P(x, y)) && in(h2, P(x, -y)); }
}
// 【凸包交】
namespace ConvecIntersection{ // ?
	const int N = 1005;
	struct Rec {
		P d[10];int dn;// d[dn] = d[0]
		P operator [] (const int&n) {return d[n];}
	}r[N];
	typedef pair<db,int> pdi;
	int n;pdi res[1000005];
	db getLoc(P a,P b,P p){
		if(sgn(b.x - a.x)) return (p.x - a.x) / (b.x - a.x);
		return (p.y - a.y) / (b.y - a.y);
	}
	db work() {
		db rt=0;
		rep(i,0,n) rep(j,0,r[i].dn){
			int sz=0;
			res[sz++] = pdi(0,0);res[sz++] = pdi(1,0);
			rep(t,0,n) {
				if(t == i) continue;
				rep(g,0,r[t].dn) {
					int du = sgn((r[i][j+1] - r[i][j]) / (r[t][g] - r[i][j]));
					int dv = sgn((r[i][j+1] - r[i][j]) / (r[t][g+1] - r[i][j]));
					if(!du && !dv) {
						if(sgn((r[i][j+1] - r[i][j]) * (r[t][g+1] - r[t][g])) < 0 || i < t){
							res[sz++] = pdi(getLoc(r[i][j] , r[i][j+1] , r[t][g]) , 1);
							res[sz++] = pdi(getLoc(r[i][j] , r[i][j+1] , r[t][g+1]) , -1);
						}} else {
							db s1 = (r[i][j] - r[t][g]) / (r[t][g+1] - r[t][g]);
							db s2 = (r[t][g+1] - r[t][g]) / (r[i][j+1] - r[t][g]);
							if(du >= 0 && dv < 0) res[sz++] = pdi(s1 / (s1 + s2) , 1);
							else if(du < 0 && dv >= 0) res[sz++] = pdi(s1 / (s1 + s2) , -1);
						}}}
			sort(res , res + sz);
			int cnt = 0; --sz;
			rep(t,0,sz) {
				cnt += res[t].se;
				if(cnt == 0 && sgn(res[t].fi - res[t+1].fi)) {
					db a = res[t].fi;
					if(a < 0) a = 0; if(a > 1) break;
					db b = res[t+1].fi;
					if(b < 0) continue; if(b > 1) b = 1;
					rt += ((r[i][j+1] - r[i][j]) * a + r[i][j]) / ((r[i][j+1]-r[i][j]) * b + r[i][j]);
				}}}
		return rt / 2;}}
\end{lstlisting}
\subsection{三角形}
\begin{lstlisting}[language=C++]
// 【心】
P outC(P A, P B, P C) { // 外心
	P b = B - A, c = C - A;
	db dB = b.len2(), dC = c.len2(), d = 2 * det(b, c);
	return A - P(b.y * dC - c.y * dB, c.x * dB - b.x * dC) / d;
}
P baryC(P p[], int n) { // 重心
	P fz(0, 0); db fm = 0;
	rep(i, 1, n - 1) {
		db t = det(p[0], p[i], p[i + 1]);
		fm += t;
		fz = fz + (p[0] + p[i] + p[i + 1]) * t / 3;
	}
	return fz / fm;
}
\end{lstlisting}
\subsection{多边形}
\begin{lstlisting}[language=C++]
// 【平面图欧拉定理】 V + F - E = 2
// 【简单多边形求面积交】
db polyInter(vector<P> &p, vector<P> &q) {
	int n = sz(p), m = sz(q);
	if(n < 3 || m < 3) return 0;
	//	if(area(p) < 0) reverse(all(p));
	//	if(area(q) < 0) reverse(all(q));
	db ans = 0;
	rep(i, 1, n - 1) {
		P p1 = p[i], p2 = p[i + 1];
		bool f1 = 0;
		if(det(p[0], p1, p2) < 0) swap(p1, p2), f1 = 1;
		rep(j, 1, m - 1) {
			P q1 = q[j], q2 = q[j + 1];
			bool f2 = 0;
			if(det(q[0], q1, q2) < 0) swap(q1, q2), f2 = 1;
			vector<P> ps({p[0], p1, p2});
			convexCut(ps, L(q[0], q1));
			convexCut(ps, L(q1, q2));
			convexCut(ps, L(q2, q[0]));
			db res = f1 == f2 ? area(ps) : -area(ps);
			ans += res;
		}
	}
	return fabs(ans);
}
\end{lstlisting}
\subsection{圆}
\begin{lstlisting}[language=C++]
// 【两圆关系】
// 注意相等关系
// 4:相离 3:外切 2:相交 1:内切 0:内含
int relCC(C A, C B) { // 两圆关系
	db dis = (A.o - B.o).len();
	if(sign(dis - (A.r + B.r)) == 1) return 4;
	if(sign(dis - (A.r + B.r)) == 0) return 3;
	if(sign(dis - fabs(A.r - B.r)) == 1) return 2;
	if(sign(dis - fabs(A.r - B.r)) == 0) return 1;
	return 0;
}
// 【点圆切点】
bool tanCP(O c, P p0, P &p1, P &p2) {
	db x = (p0 - c.o).len2(), d = x - c.r * c.r;
	if(d < eps) return 0;
	P p = (p0 - c.o) * (c.r * c.r / x);
	P det = ((p0 - c.o) * (-c.r * sqrt(d) / x)).rot90();
	p1 = c.o + p + det;
	p2 = c.o + p - det;
	return 1;
}
// 【圆圆切点】
vector<P> tanCC(const C &c1, const C &c2) {
	vector<P> res;
	db dis = (c1.o - c2.o).len();
	if(sign(dis - (c1.r + c2.r)) == 0) {
		res.pb(c1.o + (c2.o - c1.o) * c1.r / (c1.r + c2.r));
	}
	if(sign(dis - fabs(c1.r - c2.r) == 0)) {
		res.pb(c1.o + (c2.o - c1.o) * c1.r / (c1.r - c2.r));
	}
	return res;
}
// 【直线和圆求交】
bool isCL(O a, L l, P &p1, P &p2) {
  db x = dot(l.a - a.o, l.b - l.a);
  db y = (l.b - l.a).len2();
  db d = x * x - y * ((l.a - a.o).len2() - a.r * a.r);
  if(sign(d) < 0) return 0;
  d = max(d, 0.);
  P p = l.a - ((l.b - l.a) * (x / y)), det = (l.b - l.a) * (sqrt(d) / y);
  p1 = p - det, p2 = p + det; // dir : l.a -> l.b
  return 1;
}
// 【圆与三角形交面积】
db areaCT(db r,P s,P t) { // 需要除 2
	P p1, p2;
	bool f = isCL(C(P(0, 0), r), L(s, t), p1, p2);
	if(!f) return r * r * rad(s, t);
	bool b1 = sign(s.len2() - r * r) == 1 , b2 = sign(t.len2() - r * r) == 1;
	if(b1 && b2) {
		if(sign(dot(s - p1, t - p1)) <= 0 && sign(dot(s - p2, t - p2) <= 0))
			return r * r * (rad(s, p1) + rad(p2, t)) + det(p1, p2);
		else return r * r * rad(s, t);
	} else if(b1) return r * r * rad(s, p1) + det(p1, t);
	else if(b2) return r * r * rad(p2, t) + det(s, p2);
	return det(s, t);
}
// 【圆与多边形交面积】
db areaCPoly(C c, vector<P> p) { 
	int n = sz(p);
	db ans = 0;
	rep(i, 0, n) {
		P u = p[i], v = p[(i + 1) % n];
		ans += areaCT(c.r, u - c.o, v - c.o);
	}
	return fabs(ans) / 2;
}
// 【圆交】
namespace CircleIntersection{ // ?
	struct E{
		P p;T ang;int delta;
		E(){} E(P p,T ang,int delta):p(p),ang(ang),delta(delta){}
		bool operator < (const E&b) const {return ang<b.ang;}
	};
	bool overlap(C a,C b) {return sgn(a.r-b.r-abs(a.o-b.o))>=0;}
	void solve(C *c,int n,T *ans) {
		memset(ans , 0 , sizeof(T) * (n + 1));
		rep(i,0,n) {
			int cnt=1;
			vector<E> evt;
			rep(j,0,i) if(c[i]==c[j]) cnt++;
			rep(j,0,n) if(j!=i&&!(c[i]==c[j])&&overlap(c[j],c[i])) cnt++;
			rep(j,0,n) if(j!=i){
				vector<P> pts=insCC(c[i],c[j]);
				if(sz(pts)) {
					T a[2];
					rep(j,0,2) a[j]=(pts[j]-c[i].o).arg();
					evt.pb(E(pts[0],a[0],1));
					evt.pb(E(pts[1],a[1],-1));
					cnt += a[0] > a[1];
				}
			}
			if(!sz(evt)) ans[cnt] += pi*c[i].r*c[i].r;
			else{
				sort(all(evt));
				evt.pb(evt.front());
				rep(j,0,sz(evt)-1) {
					cnt+=evt[j].delta;
					ans[cnt] += evt[j].p / evt[j+1].p / 2;
					db ang = evt[j + 1].ang - evt[j].ang;
					if(ang < 0) ang += pi * 2;
					ans[cnt] += ang * c[i].r * c[i].r / 2 - sin(ang) * c[i].r * c[i].r / 2;
				}}}}}
\end{lstlisting}
\subsection{3D}
\begin{lstlisting}[language=C++]
// 【最小球覆盖】
P3 MinSphere(vector<P3> p) {
  int n = sz(p); assert(n);
  db t = 1; P3 ans(0, 0, 0);
  rep(i, 0, n) ans = ans + p[i]; ans = ans / n;
  while(t > eps) {
    int j = -1; db ret = -1;
    rep(i, 0, n) {
      db tmp = (p[i] - ans).len();
      if(ret < tmp) ret = tmp, j = i;
    }
    ans = ans + (p[j] - ans) * t;
    t *= 0.999;
  }
  return ans;
}
// 【三维向量变换】
struct Mat {
  db a[4][4];
  void set() { rep(i, 0, 4) rep(j, 0, 4) a[i][j] = 0; }
  void e() { rep(i, 0, 4) a[i][i] = 1; }
  Mat operator * (const Mat &c) {
    Mat r; r.set();
    rep(i, 0, 4) rep(j, 0, 4) rep(k, 0, 4) r.a[i][j] += a[i][k] * c.a[k][j];
    return r;
  }
};
Mat kpow(Mat a, int b) {
  Mat r; r.set(); r.e();
  while(b) {
    if(b & 1) r = r * a;
    a = a * a;
    b >>= 1;
  }
  return r;
}
Mat translate(db tx, db ty, db tz) { // 平移,以下矩阵均为左乘
  db p[4][4] = {
    1, 0, 0, tx, 
    0, 1, 0, ty,
    0, 0, 1, tz,
    0, 0, 0, 1};
  Mat r; rep(i, 0, 4) rep(j, 0, 4) r.a[i][j] = p[i][j]; return r;
}
Mat scale(db a, db b, db c) { // 缩放
  db p[4][4] = {
    a, 0, 0, 0,
    0, b, 0, 0,
    0, 0, c, 0,
    0, 0, 0, 1};
  Mat r; rep(i, 0, 4) rep(j, 0, 4) r.a[i][j] = p[i][j]; return r;
}
Mat rotate(P3 s, db a) { // 绕 s 为轴旋转 a 度,右手方向
  db l = s.len(), x = s.x / l, y = s.y / l, z = s.z / l, si = sin(a), co = cos(a);
  db p[4][4] = {
    co + (1 - co) * x * x, (1 - co) * x * y - si * z, (1 - co) * x * z + si * y, 0,
    (1 - co) * y * x + si * z, co + (1 - co) * y * y, (1 - co) * y * z - si * x, 0,
    (1 - co) * z * x - si * y, (1 - co) * z * y + si * x, co + (1 - co) * z * z, 0,
    0, 0, 0, 1};
  Mat r; rep(i, 0, 4) rep(j, 0, 4) r.a[i][j] = p[i][j]; return r;
}
\end{lstlisting}
\subsection{1、基础点、向量}
\begin{lstlisting}[language=C++]
struct P {
	int quad() const { return sign(y) > 0 || (sign(y) == 0 && sign(x) >= 0); }
	P rot90() { return P(-y, x); }
	P rot(db a) { return P(cos(a) * x - sin(a) * y, cos(a) * y + sin(a) * x); }
	P norm() { return *this / len(); }
};
db rad(P p1, P p2) { return atan2l(det(p1, p2), dot(p1, p2)); } // p1 与 p2 的夹角,有方向
bool cmp(const pii &a, const pii &b) { // 级角排序
	int o = a > pii(0, 0), t = b > pii(0, 0);
	if(o != t) return o < t;
	return det(a, b) > 0;
}
// 【点集中最近点对】
namespace NearestPoints { // sz(A) <= 1e5
	db solve(int l, int r, vector<P> &p) {
		if(l == r) return 1e100;
		int m = l + r >> 1;
		db Xm = p[m].x, lim = min(solve(l, m, p), solve(m + 1, r, p));
		inplace_merge(p.begin() + l, p.begin() + m + 1, p.begin() + r + 1, [&](P a, P b){return a.y < b.y;});
		vector<P> V;
		rep(i, l, r + 1) if(fabs(p[i].x - Xm) <= lim) V.pb(p[i]);
		rep(i, 0, sz(V)) rep(j, i + 1, sz(V)) {
			if(fabs(V[j].y - V[i].y) >= lim) break;
			T dis = (V[i] - V[j]).len();
			lim = min(lim, dis);
		}
		return lim;
	}
	db solve(vector<P> A) {
		sort(all(A), [&](P a, P b){return a.x < b.x;});
		return solve(0, sz(A) - 1, A);
	}
}
// 【最小圆覆盖】
C Mincir(P *p,int n){ 
	random_shuffle(p , p + n);
	P o = p[0];db r = 0;
	rep(i,1,n) {
		if(sgn(abs(o-p[i])-r) <= 0) continue;
		o = p[i] , r = 0;
		rep(j,0,i) {
			if(sgn(abs(o-p[j])-r) <= 0) continue;
			o = (p[i] + p[j]) / 2 , r = abs(o-p[j]);
			rep(k,0,j) {
				if(sgn(abs(o-p[k])-r) <= 0) continue;
				o = outC(p[i],p[j],p[k]) , r = abs(o-p[k]);
			}}}
	return C(o,r);
}
// 【费马点】
// sqrt((a ^ 2 + b ^ 2 + c ^ 2 + 4 * sqrt(3) * area) / 2)
// 如果有重点,大于 2 的直接用模拟退火法
P fermat(vector<P> p) {
  int n = sz(p); assert(n);
  if(n == 1) return p[0];
  if(n == 2) return (p[0] + p[1]) / 2;
  if(n == 3) {
    db a[3];
    rep(i, 0, 3) a[i] = (p[(i + 2) % 3] - p[(i + 1) % 3]).len();
    rep(i, 0, 3) {
      int j = (i + 1) % 3, k = (i + 2) % 3;
      if(sign(a[i] * a[i] - a[j] * a[j] - a[k] * a[k] - a[j] * a[k]) >= 0) return p[i];
    }
    if(det(p[0], p[1], p[2]) < 0) swap(p[1], p[2]);
    P q1 = (p[2] - p[0]).rot(pi / 3) + p[0]; 
    P q2 = (p[0] - p[1]).rot(pi / 3) + p[1];
    return isLL(L(q1, p[1]), L(q2, p[2]));
  }
  auto Rand = [&] () { return rand() % 10000 / 5000 * pi; };
  P ans(0, 0); rep(i, 0, n) ans = ans + p[i]; ans = ans / n;
  db len = 0; rep(i, 0, n) len += (ans - p[i]).len();
  db t = 10000; // modify
  while(t > eps) {
    db ang = Rand();
    P np(ans.x + t * sin(ang), ans.y + t * cos(ang));
    db k = 0; rep(i, 0, n) k += (np - p[i]).len();
    if(sign(len - k) > 0) ans = np, len = k;
    t *= 0.999;
  }
  return ans;
}
\end{lstlisting}
\subsection{2、线段、直线、曲线}
\begin{lstlisting}[language=C++]
// 【直线交点】
P isLL(L l1, L l2) {
	db s1 = det(l2.b - l2.a, l1.a - l2.a);
	db s2 = -det(l2.b - l2.a, l1.b - l2.a);
	return (l1.a * s2 + l1.b * s1) / (s1 + s2);
}
P isLL(L l, db a, db b, db c) { // ax + by + c = 0
	db u = a * l.a.x + b * l.a.y + c;
	db v = -(a * l.b.x + b * l.b.y + c);
	return (l.a * v + l.b * u) / (u + v);
}
P isLL(db a0, db b0, db c0, db a1, db b1, db c1) {
	db d = a0 * b1 - a1 * b0;
	return P(b0 * c1 - b1 * c0, a1 * c0 - a0 * c1) / d;
}
// 【线相交判定】
bool isSSr(const L &a, const L &b){
	db c1 = det(a.t - a.s, b.s - a.s), c2 = det(a.t - a.s, b.t - a.s);
	db c3 = det(b.t - b.s, a.s - b.s), c4 = det(b.t - b.s, a.t - b.s);
	return sign(c1) * sign(c2) < 0 && sign(c3) * sign(c4) < 0;
}
bool isSS(L a,L b){
	db c1 = det(a.t - a.s, b.s - a.s), c2 = det(a.t - a.s, b.t - a.s);
	db c3 = det(b.t - b.s, a.s - b.s), c4 = det(b.t - b.s, a.t - b.s);
	return sign(c1) * sign(c2) <= 0 && sign(c3) * sign(c4) <= 0 &&
		sign(max(a.s.x, a.t.x) - min(b.s.x, b.t.x)) >= 0 &&
		sign(max(b.s.x, b.t.x) - min(a.s.x, a.t.x)) >= 0 &&
		sign(max(a.s.y, a.t.y) - min(b.s.y, b.t.y)) >= 0 &&
		sign(max(b.s.y, b.t.y) - min(a.s.y, a.t.y)) >= 0;
}
bool isLS(P a1, P a2, P b1, P b2) { // 判断直线线段是否相交(端点也算)
	db c1 = det(a2 - a1, b1 - a1), c2 = det(a2 - a1, b2 - a1);
	return sign(c1) * sign(c2) <= 0;
}
// 【点到线距离】
db disToL(L l, P p) {
  return fabs(det(l.a, p, l.b) / (l.b - l.a).len());
}
db disToS(L l, P p) {
  return sign(dot(l.a, p, l.b)) * sign(dot(l.b, p, l.a)) == 1 ? disToL(l, p) : min((p - l.a).len(), (p - l.b).len());
}
// 【线到线距离】
db disSS(L a, L b){
	if(isSS(a, b)) return 0;
	return min(min(disToSeg(b, a.s), disToSeg(b, a.t)), min(disToSeg(a, b.s), disToSeg(a, b.t)));
}
\end{lstlisting}
\subsection{3、凸包}
\begin{lstlisting}[language=C++]
// 【求凸包】
vector<P> convexHull(vector<P> ps) {
  int n = sz(ps); if(n <= 1) return ps;
  sort(all(ps)); vector<P> qs;
  for(int i = 0; i < n; qs.pb(ps[i++])) {
    while(sz(qs) > 1 && sign(det(qs[sz(qs) - 2], qs.back(), ps[i])) <= 0) qs.pop_back();
  }
  for(int i = n - 2, t = sz(qs); i >= 0; qs.pb(ps[i--])) {
    while(sz(qs) > t && sign(det(qs[sz(qs) - 2], qs.back(), ps[i])) <= 0) qs.pop_back();
  }
  qs.pop_back(); return qs;
}
// 【凸包最远点对】
db diameter(vector<P> A) {
	int n = sz(A);
	if(n <= 1) return 0;
	int l = 0, r = 0;
	rep(i, 1, n) (A[i] < A[l]) && (l = i), (A[r] < A[i]) && (r = i);
	db res = (A[l]-A[r]).len();
	int i = l, j = r;
	do (++(det(A[(i + 1) % n]- A[i], A[(j + 1) % n] - A[j]) >= 0 ? j : i)) %= n,
		res = max(res, (A[i] - A[j]).len());
	while(i != l || j != r);
	return res;
}
// 【动态凸包】
// O(nlogn)
// 插入点,询问点在不在凸包内部(包括边界)
namespace DCH {
	map<int, P> h1, h2;
	bool ao(P a, P b, P c) {
		// 包括边界:小等于
		return (b.y - a.y) * 1ll * (c.x - b.x) <= (c.y - b.y) * 1ll * (b.x - a.x);
	}
	bool in(map<int, P> &h, P p) {
		if(!sz(h)) return 0;
		if(p.x < h.begin()->se.x || p.x > h.rbegin()->se.x) return 0;
		auto l = h.lower_bound(p.x);
		if(p.x == l->se.x) return p.y <= l->se.y;
		auto r = l--;
		return ao(l->se, p, r->se);
	}
	void ins(map<int, P> &h, P p) {
		if(in(h, p)) return ;
		h[p.x] = p;
		auto pos = h.find(p.x);
		while(1) {
			auto l = pos; if(l == h.begin()) break; --l;
			auto ll = l; if(ll == h.begin()) break; --ll;
			if(ao(ll->se, l->se, p)) h.erase(l); else break;
		}
		while(1) {
			auto r = pos; r++; if(r == h.end()) break;
			auto rr = r; rr++; if(rr == h.end()) break;
			if(ao(p, r->se, rr->se)) h.erase(r); else break;
		}
	}
	void ins(int x, int y) { ins(h1, P(x, y)); ins(h2, P(x, -y)); }
	bool in(int x, int y) { return in(h1, P(x, y)) && in(h2, P(x, -y)); }
}
// 【凸包交】
namespace ConvecIntersection{ // ?
	const int N = 1005;
	struct Rec {
		P d[10];int dn;// d[dn] = d[0]
		P operator [] (const int&n) {return d[n];}
	}r[N];
	typedef pair<db,int> pdi;
	int n;pdi res[1000005];
	db getLoc(P a,P b,P p){
		if(sgn(b.x - a.x)) return (p.x - a.x) / (b.x - a.x);
		return (p.y - a.y) / (b.y - a.y);
	}
	db work() {
		db rt=0;
		rep(i,0,n) rep(j,0,r[i].dn){
			int sz=0;
			res[sz++] = pdi(0,0);res[sz++] = pdi(1,0);
			rep(t,0,n) {
				if(t == i) continue;
				rep(g,0,r[t].dn) {
					int du = sgn((r[i][j+1] - r[i][j]) / (r[t][g] - r[i][j]));
					int dv = sgn((r[i][j+1] - r[i][j]) / (r[t][g+1] - r[i][j]));
					if(!du && !dv) {
						if(sgn((r[i][j+1] - r[i][j]) * (r[t][g+1] - r[t][g])) < 0 || i < t){
							res[sz++] = pdi(getLoc(r[i][j] , r[i][j+1] , r[t][g]) , 1);
							res[sz++] = pdi(getLoc(r[i][j] , r[i][j+1] , r[t][g+1]) , -1);
						}} else {
							db s1 = (r[i][j] - r[t][g]) / (r[t][g+1] - r[t][g]);
							db s2 = (r[t][g+1] - r[t][g]) / (r[i][j+1] - r[t][g]);
							if(du >= 0 && dv < 0) res[sz++] = pdi(s1 / (s1 + s2) , 1);
							else if(du < 0 && dv >= 0) res[sz++] = pdi(s1 / (s1 + s2) , -1);
						}}}
			sort(res , res + sz);
			int cnt = 0; --sz;
			rep(t,0,sz) {
				cnt += res[t].se;
				if(cnt == 0 && sgn(res[t].fi - res[t+1].fi)) {
					db a = res[t].fi;
					if(a < 0) a = 0; if(a > 1) break;
					db b = res[t+1].fi;
					if(b < 0) continue; if(b > 1) b = 1;
					rt += ((r[i][j+1] - r[i][j]) * a + r[i][j]) / ((r[i][j+1]-r[i][j]) * b + r[i][j]);
				}}}
		return rt / 2;}}
\end{lstlisting}
\subsection{4、三角形}
\begin{lstlisting}[language=C++]
// 【心】
P outC(P A, P B, P C) { // 外心
	P b = B - A, c = C - A;
	db dB = b.len2(), dC = c.len2(), d = 2 * det(b, c);
	return A - P(b.y * dC - c.y * dB, c.x * dB - b.x * dC) / d;
}
P baryC(P p[], int n) { // 重心
	P fz(0, 0); db fm = 0;
	rep(i, 1, n - 1) {
		db t = det(p[0], p[i], p[i + 1]);
		fm += t;
		fz = fz + (p[0] + p[i] + p[i + 1]) * t / 3;
	}
	return fz / fm;
}
\end{lstlisting}
\subsection{5、多边形}
\begin{lstlisting}[language=C++]
// 【平面图欧拉定理】 V + F - E = 2
// 【简单多边形求面积交】
db polyInter(vector<P> &p, vector<P> &q) {
	int n = sz(p), m = sz(q);
	if(n < 3 || m < 3) return 0;
	//	if(area(p) < 0) reverse(all(p));
	//	if(area(q) < 0) reverse(all(q));
	db ans = 0;
	rep(i, 1, n - 1) {
		P p1 = p[i], p2 = p[i + 1];
		bool f1 = 0;
		if(det(p[0], p1, p2) < 0) swap(p1, p2), f1 = 1;
		rep(j, 1, m - 1) {
			P q1 = q[j], q2 = q[j + 1];
			bool f2 = 0;
			if(det(q[0], q1, q2) < 0) swap(q1, q2), f2 = 1;
			vector<P> ps({p[0], p1, p2});
			convexCut(ps, L(q[0], q1));
			convexCut(ps, L(q1, q2));
			convexCut(ps, L(q2, q[0]));
			db res = f1 == f2 ? area(ps) : -area(ps);
			ans += res;
		}
	}
	return fabs(ans);
}
\end{lstlisting}
\subsection{6、圆}
\begin{lstlisting}[language=C++]
// 【两圆关系】
// 注意相等关系
// 4:相离 3:外切 2:相交 1:内切 0:内含
int relCC(C A, C B) { // 两圆关系
	db dis = (A.o - B.o).len();
	if(sign(dis - (A.r + B.r)) == 1) return 4;
	if(sign(dis - (A.r + B.r)) == 0) return 3;
	if(sign(dis - fabs(A.r - B.r)) == 1) return 2;
	if(sign(dis - fabs(A.r - B.r)) == 0) return 1;
	return 0;
}
// 【点圆切点】
bool tanCP(O c, P p0, P &p1, P &p2) {
	db x = (p0 - c.o).len2(), d = x - c.r * c.r;
	if(d < eps) return 0;
	P p = (p0 - c.o) * (c.r * c.r / x);
	P det = ((p0 - c.o) * (-c.r * sqrt(d) / x)).rot90();
	p1 = c.o + p + det;
	p2 = c.o + p - det;
	return 1;
}
// 【圆圆切点】
vector<P> tanCC(const C &c1, const C &c2) {
	vector<P> res;
	db dis = (c1.o - c2.o).len();
	if(sign(dis - (c1.r + c2.r)) == 0) {
		res.pb(c1.o + (c2.o - c1.o) * c1.r / (c1.r + c2.r));
	}
	if(sign(dis - fabs(c1.r - c2.r) == 0)) {
		res.pb(c1.o + (c2.o - c1.o) * c1.r / (c1.r - c2.r));
	}
	return res;
}
// 【直线和圆求交】
bool isCL(O a, L l, P &p1, P &p2) {
  db x = dot(l.a - a.o, l.b - l.a);
  db y = (l.b - l.a).len2();
  db d = x * x - y * ((l.a - a.o).len2() - a.r * a.r);
  if(sign(d) < 0) return 0;
  d = max(d, 0.);
  P p = l.a - ((l.b - l.a) * (x / y)), det = (l.b - l.a) * (sqrt(d) / y);
  p1 = p - det, p2 = p + det; // dir : l.a -> l.b
  return 1;
}
// 【圆与三角形交面积】
db areaCT(db r,P s,P t) { // 需要除 2
	P p1, p2;
	bool f = isCL(C(P(0, 0), r), L(s, t), p1, p2);
	if(!f) return r * r * rad(s, t);
	bool b1 = sign(s.len2() - r * r) == 1 , b2 = sign(t.len2() - r * r) == 1;
	if(b1 && b2) {
		if(sign(dot(s - p1, t - p1)) <= 0 && sign(dot(s - p2, t - p2) <= 0))
			return r * r * (rad(s, p1) + rad(p2, t)) + det(p1, p2);
		else return r * r * rad(s, t);
	} else if(b1) return r * r * rad(s, p1) + det(p1, t);
	else if(b2) return r * r * rad(p2, t) + det(s, p2);
	return det(s, t);
}
// 【圆与多边形交面积】
db areaCPoly(C c, vector<P> p) { 
	int n = sz(p);
	db ans = 0;
	rep(i, 0, n) {
		P u = p[i], v = p[(i + 1) % n];
		ans += areaCT(c.r, u - c.o, v - c.o);
	}
	return fabs(ans) / 2;
}
// 【圆交】
namespace CircleIntersection{ // ?
	struct E{
		P p;T ang;int delta;
		E(){} E(P p,T ang,int delta):p(p),ang(ang),delta(delta){}
		bool operator < (const E&b) const {return ang<b.ang;}
	};
	bool overlap(C a,C b) {return sgn(a.r-b.r-abs(a.o-b.o))>=0;}
	void solve(C *c,int n,T *ans) {
		memset(ans , 0 , sizeof(T) * (n + 1));
		rep(i,0,n) {
			int cnt=1;
			vector<E> evt;
			rep(j,0,i) if(c[i]==c[j]) cnt++;
			rep(j,0,n) if(j!=i&&!(c[i]==c[j])&&overlap(c[j],c[i])) cnt++;
			rep(j,0,n) if(j!=i){
				vector<P> pts=insCC(c[i],c[j]);
				if(sz(pts)) {
					T a[2];
					rep(j,0,2) a[j]=(pts[j]-c[i].o).arg();
					evt.pb(E(pts[0],a[0],1));
					evt.pb(E(pts[1],a[1],-1));
					cnt += a[0] > a[1];
				}
			}
			if(!sz(evt)) ans[cnt] += pi*c[i].r*c[i].r;
			else{
				sort(all(evt));
				evt.pb(evt.front());
				rep(j,0,sz(evt)-1) {
					cnt+=evt[j].delta;
					ans[cnt] += evt[j].p / evt[j+1].p / 2;
					db ang = evt[j + 1].ang - evt[j].ang;
					if(ang < 0) ang += pi * 2;
					ans[cnt] += ang * c[i].r * c[i].r / 2 - sin(ang) * c[i].r * c[i].r / 2;
				}}}}}
\end{lstlisting}
\subsection{7、3D}
\begin{lstlisting}[language=C++]
// 【最小球覆盖】
P3 MinSphere(vector<P3> p) {
  int n = sz(p); assert(n);
  db t = 1; P3 ans(0, 0, 0);
  rep(i, 0, n) ans = ans + p[i]; ans = ans / n;
  while(t > eps) {
    int j = -1; db ret = -1;
    rep(i, 0, n) {
      db tmp = (p[i] - ans).len();
      if(ret < tmp) ret = tmp, j = i;
    }
    ans = ans + (p[j] - ans) * t;
    t *= 0.999;
  }
  return ans;
}
// 【三维向量变换】
struct Mat {
  db a[4][4];
  void set() { rep(i, 0, 4) rep(j, 0, 4) a[i][j] = 0; }
  void e() { rep(i, 0, 4) a[i][i] = 1; }
  Mat operator * (const Mat &c) {
    Mat r; r.set();
    rep(i, 0, 4) rep(j, 0, 4) rep(k, 0, 4) r.a[i][j] += a[i][k] * c.a[k][j];
    return r;
  }
};
Mat kpow(Mat a, int b) {
  Mat r; r.set(); r.e();
  while(b) {
    if(b & 1) r = r * a;
    a = a * a;
    b >>= 1;
  }
  return r;
}
Mat translate(db tx, db ty, db tz) { // 平移,以下矩阵均为左乘
  db p[4][4] = {
    1, 0, 0, tx, 
    0, 1, 0, ty,
    0, 0, 1, tz,
    0, 0, 0, 1};
  Mat r; rep(i, 0, 4) rep(j, 0, 4) r.a[i][j] = p[i][j]; return r;
}
Mat scale(db a, db b, db c) { // 缩放
  db p[4][4] = {
    a, 0, 0, 0,
    0, b, 0, 0,
    0, 0, c, 0,
    0, 0, 0, 1};
  Mat r; rep(i, 0, 4) rep(j, 0, 4) r.a[i][j] = p[i][j]; return r;
}
Mat rotate(P3 s, db a) { // 绕 s 为轴旋转 a 度,右手方向
  db l = s.len(), x = s.x / l, y = s.y / l, z = s.z / l, si = sin(a), co = cos(a);
  db p[4][4] = {
    co + (1 - co) * x * x, (1 - co) * x * y - si * z, (1 - co) * x * z + si * y, 0,
    (1 - co) * y * x + si * z, co + (1 - co) * y * y, (1 - co) * y * z - si * x, 0,
    (1 - co) * z * x - si * y, (1 - co) * z * y + si * x, co + (1 - co) * z * z, 0,
    0, 0, 0, 1};
  Mat r; rep(i, 0, 4) rep(j, 0, 4) r.a[i][j] = p[i][j]; return r;
}
\end{lstlisting}
\subsection{HalfPlane\_n2}
\begin{lstlisting}[language=C++]
// l: a->b 逆时针方向  
void convexCut(vector<P> &p, L l) {
	vector<P> q;
	rep(i, 0, sz(p)) {
		P p1 = p[i], p2 = p[(i + 1) % sz(p)];
		int d1 = sign(det(l.a, l.b, p1));
		int d2 = sign(det(l.a, l.b, p2));
		if(d1 >= 0) q.pb(p1);
		if(d1 * d2 < 0) q.pb(isLL(L(p1, p2), l));
	} p = q;
}
// ax + by + c >= 0
void convexCut(vector<P> &p, db a, db b, db c) {
	vector<P> q;
	rep(i, 0, sz(p)) {
		P p1 = p[i], p2 = p[(i + 1) % sz(p)];
		int d1 = sign(a * p1.x + b * p1.y + c);
		int d2 = sign(a * p2.x + b * p2.y + c);
		if(d1 >= 0) q.pb(p1);
		if(d1 * d2 < 0) q.pb(isLL(L(p1, p2), a, b, c));
	} p = q;
}
\end{lstlisting}
\subsection{HalfPlane\_nlogn}
\begin{lstlisting}[language=C++]
struct P {
	int quad() const { return sign(y) > 0 || (sign(y) == 0 && sign(x) >= 0); }
};
struct L {
	// ax + by + c >= 0, (a != 0 || b != 0)
	L(db a, db b, db c) { 
		if(sign(a)==0) {
			this->a=P(0,-c/b);this->b=P(sign(b),-c/b);
		} else if(sign(b)==0) {
			this->a=P(-c/a,0);this->b=P(-c/a,-sign(a));
		} else {
			if(sign(c)!=0) {
				int x=sign(c)*sign(det(P(-c/a,0), P(0,-c/b)));
				if(x==1) this->a=P(-c/a,0),this->b=P(0,-c/b);
				else this->a=P(0,-c/b),this->b=P(-c/a,0);
			} else {
				this->a=P(0,0);this->b=P(sign(b),sign(b)*(-a/b));
			}
		}
	}
	bool includer(const P &p) const { return sign(det(b - a, p - a)) > 0; }
	bool include(const P &p) const { return sign(det(b - a, p - a)) >= 0; }
	// 向内(右手方向)推
	L push(db len) {
		P det = (b - a).rot90().norm() * len;
		return L(a + det, b + det);
	}
};
bool sameDir(L l0, L l1) {
	P a = l0.a - l0.b, b = l1.a - l1.b;
	return sign(det(a, b)) == 0 && sign(dot(a, b)) == 1;
}
bool operator < (const P &a,  const P &b) {
	if(a.quad() != b.quad()) return a.quad() < b.quad();
	return sign(det(a, b)) > 0;
}
bool operator < (const L &l0, const L &l1) {
	if(sameDir(l0, l1)) return l1.includer(l0.a);
	return (l0.b - l0.a) < (l1.b - l1.a);
}
bool check(L u, L v, L w) { return w.include(isLL(u, v)); }
deque<L> halfPlane(vector<L> l) {
	sort(all(l)); deque<L> q;
	rep(i, 0, sz(l)) {
		if(i && sameDir(l[i], l[i - 1])) continue;
		while(sz(q) > 1 && !check(q[sz(q) - 2], q.back(), l[i])) q.pop_back();
		while(sz(q) > 1 && !check(q[1], q[0], l[i])) q.pop_front();
		q.pb(l[i]);
	}
	while(sz(q) > 2 && !check(q[sz(q) - 2], q.back(), q[0])) q.pop_back();
	while(sz(q) > 2 && !check(q[1], q[0], q.back())) q.pop_front();
	return q;
}
\end{lstlisting}
\subsection{MaxAreaPoly}
\begin{lstlisting}[language=C++]
ld solve_poly(vi &S) {
	assert(sz(S) > 0);
	int sum = 0, hi = S[0];
	vi vals;
	rep(i, 1, sz(S)) {
		int cur = S[i];
		if (cur > hi) swap(cur, hi);
		sum += cur;
		vals.pb(cur);
	}
	if (sum <= hi) return 0;
	auto getAngle = [&](ld D) -> ld{
		ld tot = 0;
		for (int l : vals) tot += 2 * asin(ld(l) / ld(D));
		return tot;
	};
	bool isReflex = (getAngle(hi) < PI);
	auto tooSmall = [&](ld D) {
		ld ang = getAngle(D);
		ld hiAng = 2 * asin(ld(hi) / ld(D));
		if (isReflex) return ang < hiAng;
		else return ang + hiAng >= 2 * PI;
	};
	ld mi = hi, ma = hi + 1;
	int numExpand = 0;
	while (tooSmall(ma)) numExpand++, ma += (ma - mi);
	rep(tim, 0, 50 + numExpand) {
		ld md = mi + (ma - mi) / 2;
		if (tooSmall(md)) mi = md;
		else ma = md;
	}
	ld D = mi, area = 0;
	for (int l : vals) area += ld(l) * sqrt(ld(D) * ld(D) - ld(l) * ld(l)) / 4;
	ld hiArea = ld(hi) * sqrt(ld(D) * ld(D) - ld(hi) * ld(hi)) / 4;
	if (isReflex) area -= hiArea;
	else area += hiArea;
	return area;
}
\end{lstlisting}
\subsection{MaxAreaTri}
\begin{lstlisting}[language=C++]
// O(n ^ 2)
void maxAreaTri(P *p, int n, P &a, P &b, P &c) {
	int i = 0, j = 1, k = 2;
	a = p[i], b = p[j], c = p[k];
	T res = area(a, b, c), cur = res, tmp;
	do {
		while(1) {
			while(cur <= (tmp = area(p[i], p[j], p[(k + 1) % n]))) (++k) %= n, cur = tmp;
			if(cur <= (tmp = area(p[i], p[(j + 1) % n], p[k]))) (++j) %= n, cur = tmp;
			else break;
		}
		if(cur > res) a = p[i], b = p[j], c = p[k], res = cur;
		(++i) %= n;
		if(i == j) (++j) %= n;
		if(j == k) (++k) %= n;
		cur = area(p[i], p[j], p[k]);
	} while(i);
}
\end{lstlisting}
\subsection{MinAreaTri}
\begin{lstlisting}[language=C++]
// 无重点、三点共线
// O(n^2log_2n)
struct P { int x, y, ind, u, v; };
namespace MinAreaTri {
	const int N = 2020;
	const ll inf = 4e18;
	int n, m, pos[N];
	P p[N], l[N * N];
	bool cmp(const P &x, const P &y) { return det(x, y) < 0; }
	void solve() {
		sort(p + 1, p + 1 + n);
		rep(i, 1, n + 1) p[i].ind = i, pos[i] = i;
		m = 0; rep(i, 1, n + 1) rep(j, i + 1, n + 1) {
			l[++m] = p[i] - p[j];
			if(l[m].x < 0) l[m].x *= -1, l[m].y *= -1;
			else if(l[m].x == 0 && l[m].y < 0) l[m].y *= -1;
			l[m].u = i, l[m].v = j;
		}
		sort(l + 1, l + 1 + m, cmp);
		mi = inf, ma = 0;
		rep(i, 1, m + 1) {
			int u = l[i].u, v = l[i].v;
			int pu = pos[u], pv = pos[v];
			if(pu > pv) swap(u, v), swap(pu, pv);
			if(pu == 1 || pv == n) continue;
			mi = min(mi, area(p[pu - 1], p[pu], p[pv + 1], p[v]));
			ma = max(ma, area(p[1], p[pu], p[n], p[v]));
			swap(p[pu], p[pv]);
			swap(pos[u], pos[v]);
		}
		cout << mi << " " << ma << endl;
	}
}
\end{lstlisting}
\subsection{凹四边形计数}
\begin{lstlisting}[language=C++]
const int N = 1010;
int n; P p[N], q[N]; ll s[N];
namespace CNT {
	bool gao(P a) { return a.y > 0 || (a.y == 0 && a.x >= 0); }
	bool cmp(P a, P b) {
		bool o = gao(a), t = gao(b);
		if(o != t) return o > t;
		return det(a, b) > 0;
	}
	void solve(int u, ll &ans) {
		rep(i, 1, n + 1) q[i] = p[i]; swap(q[1], q[u]);
		rep(i, 2, n + 1) q[i] = q[i] - p[u];
		sort(q + 2, q + n + 1, cmp);
		int k = n; while(k >= 2 && q[k].y <= 0) --k;
		int j = k, cnt = 0;
		per(i, k + 1, n + 1) {
			while(j >= 2 && det(q[j], q[i]) > 0) --j, ++cnt;
			s[i] = s[i + 1] + cnt;
		}
		int c = j = k + 1;
		rep(i, 2, k + 1) {
			while(c <= n && det(q[i], q[c]) > 0) ++c;
			while(j <= n && det(q[i], q[j]) >= 0) ++j;
			ans += s[j] + (n - j + 1) * 1ll * (c - k - 1);
		}
	}
	ll solve() {
		ll ans = 0; rep(i, 1, n + 1) solve(i, ans);
		return ans;
	}
}
\end{lstlisting}
\subsection{平面图转对偶图}
\begin{lstlisting}[language=C++]
struct Planar {
	static const int N = 101010, M = 101010;
	// ps id starts from 0
	vector<P> ps;
	// cnte id starts from 0
	int cnte, ne[M];
	bool vis[M];
	// u -> (v, cnte)
	vector<pii> g[N];
	pii E[M];
	vector<db> areas;

	void init() {
		rep(i, 0, sz(ps)) g[i].clear();
		fill_n(vis, cnte, false);
		ps.clear(); cnte = 0;
		areas.clear();
	}
	void adde(int u, int v) {
		g[u].pb(mp(v, cnte));
		E[cnte++] = mp(u, v);
		g[v].pb(mp(u, cnte));
		E[cnte++] = mp(v, u);
	}
	int V;
	bool cmp(const pii &i, const pii &j) {
		P a = ps[i.fi] - ps[V], b = ps[j.fi] - ps[V];
		int o = P(0, 0) < a, t = P(0, 0) < b;
		if(o != t) return o < t;
		return det(a, b) > 0;
	}
	void go(int e) {
		db res = 0;
		while(!vis[e]) {
			res += det(ps[E[e].se], ps[E[e].fi]); vis[e] = 1;
			e = ne[e ^ 1];
		}
		if(res > 0) areas.pb(res / 2);
	}
	void solve(const vector<P> &_ps, const vector<pii> &es) {
		init(); ps = _ps;
		for(auto e : es) adde(e.fi, e.se);
		rep(i, 0, sz(ps)) {
			V = i; sort(all(g[i]), cmp);
			rep(j, 0, sz(g[i])) {
				ne[g[i][j].se] = g[i][(j + 1) % sz(g[i])].se;
			}
		}
		rep(i, 0, cnte) if(!vis[i]) go(i);
	}
};
\end{lstlisting}
\subsection{旋转卡壳}
\begin{lstlisting}[language=C++]
// 凸包都是顺时针给出
// 【凸包直径】 点 - 点
T diameter(vector<P> ps) {
	n = sz(ps); T ans = 0;
	if(n <= 1) return 0;
	if(n == 2) return (ps[1] - ps[0]).len();
	rep(i, 0, n) {
		P t = ps[i] - ps[(i + 1) % n];
		while(det(t, ps[(p + 1) % n] - ps[p]) > 0) (++p) %= n;
		ans = max(ans, (ps[i] - ps[p]).len());
		ans = max(ans, (ps[(i + 1) % n] - ps[p]).len());
	}
	return ans;
}
// 【凸包宽度】 点 - 边
// 【凸包间的最大距离】 点 - 点
// 【凸包间的最小距离】
T solve(P p[], int n, P q[], int m) {
	int o = 0, t = 0; T ans = inf;
	rep(i, 1, n) if(p[i].y > p[o].y) o = i;
	rep(i, 1, m) if(q[i].y < q[t].y) t = i;
	rep(i, 0, n) {
		P a = p[(o + 1) % n] - p[o]; db tmp;
		while((tmp = det(a, q[(t + 1) % m] - q[t])) < 0) (++t) %= m;
		if(sign(tmp)) ans = min(ans, disToSeg(L(p[o], p[(o + 1) % n]), q[t]));
		else ans = min(ans, disSS(L(p[o], p[(o + 1) % n]), L(q[t], q[(t + 1) % m])));
		(++o) %= n;
	}
	return ans;
}
T work(P p[], int n, P q[], int m) {
	return min(solve(p, n, q, m), solve(q, m, p, n));
}
// 【凸包最小面积外接矩形】
T solve(vector<P> ps) {
	int n = sz(ps); T ans = 1e18;
	int p = 1, l = 1, r;
	rep(i, 0, n) {
		P t = ps[i] - ps[(i + 1) % n];
		while(det(t, ps[(p + 1) % n] - ps[p]) > 0) (++p) %= n;
		while(dot(t, ps[(l + 1) % n] - ps[l]) < 0) (++l) %= n;
		r = (p + 1) % n;
		while(dot(t, ps[(r + 1) % n] - ps[r]) > 0) (++r) %= n;
		ll et = abs(det(ps[p], ps[i], ps[(i + 1) % n]));
		ll ot = abs(dot(t, ps[l] - ps[r]));
		ans = min(ans, (db)et * ot / t.len2());
	}
	return ans;
}
// 【凸包最小周长外接矩形】
\end{lstlisting}

\section{Graph}
\subsection{矩阵-树定理(生成树计数)}
\begin{lstlisting}[language=C++]
/*生成树计数问题
n 个节点的无向图 G ,求一个包含 n-1 条边的边集使得边集的边构成一颗树,问这样的边集的数量
度数矩阵 D n*n的矩阵 Di,i=i的度数 Di,j=0(i!=j)
邻接矩阵 A n*n的矩阵 Ai,j=(i与j有边相连)?1:0
基尔霍夫矩阵 M M=D-A Mi,i=i的度数 Mi,j=(i与j有边相连)?-1:0(i!=j)
矩阵树定理 对于图G,它的基尔霍夫矩阵M的每个代数余子式相等,且等于G的生成树的数目。*/
\end{lstlisting}

\section{Java}
\subsection{IO}
\begin{lstlisting}[language=C++]
package mytest;
//提交评测前删除package
import java.io.*;
import java.util.*;
import java.math.*;

public class Main {
    BufferedReader reader = new BufferedReader(new InputStreamReader(System.in));
    PrintWriter writer = new PrintWriter(System.out);
    StringTokenizer tokenizer = null;
    void solve()throws Exception{
    	List<String> mylist1 = new ArrayList<>();
		List<String> mylist2 = new LinkedList<>();
		List<String> mylist3 = new Vector<>();
		Vector<String> vec = new Vector<>();
		Queue<String> que = new LinkedList<>();
		Stack<String> sta = new Stack<>();
		Set<String> myset = new HashSet<>();
		Set<String> myset2 = new TreeSet<>();
		Map<String, Integer> mymap = new HashMap<>();
		Map<String, Integer> mymap2 = new TreeMap<>();
    }
    void run()throws Exception{
    	try {
    		while (true) {
    			solve();
    		}
    	}catch(Exception e) {}
    	finally {
    		reader.close();
    		writer.close();
    	}
    }
    String next()throws Exception{ 
    	for(;tokenizer == null || !tokenizer.hasMoreTokens();){
    		tokenizer = new StringTokenizer(reader.readLine());
    	}
    	return tokenizer.nextToken();
    }
    int nextInt()throws Exception{
    	return Integer.parseInt(next());
    }
    double nextDouble()throws Exception{
    	return Double.parseDouble(next());
    }
    BigInteger nextBigInteger()throws Exception{
    	return new BigInteger(next());
    }
    public static void main(String args[])throws Exception{
    	(new Main()).run();
    }
}
\end{lstlisting}

\section{Math}
\subsection{FFT}
\begin{lstlisting}[language=C++]
const double PI = acos(-1.0);
const int _M = N, _N = N;
template <class V>
struct FT {
	struct cp { double x, y; } tmp[_M * 2 + 5];
	friend cp operator + (cp &a, cp &b) { return cp{ a.x + b.x,a.y + b.y }; }
	friend cp operator - (cp &a, cp &b) { return cp{ a.x - b.x,a.y - b.y }; }
	friend cp operator * (cp &a, cp &b) { return cp{ a.x*b.x - a.y*b.y,a.x*b.y + a.y*b.x }; }
	cp get(double x) { return cp{ cos(x),sin(x) }; }
	vector <cp> aa, bb;
	void FFT(vector<cp> &a, int n, int op) {
		for (int i = (n >> 1), j = 1; j < n; j++) {
			if (i < j) swap(a[i], a[j]);
			int k; for (k = (n >> 1); k&i; i ^= k, k >>= 1); i ^= k;
		}
		for (int m = 2; m <= n; m <<= 1) {
			cp w = get(2 * PI*op / m); tmp[0] = cp{ 1,0 };
			for (int j = 1; j < (m >> 1); j++) tmp[j] = tmp[j - 1] * w;
			for (int i = 0; i < n; i += m)
				for (int j = i; j < i + (m >> 1); j++) {
					cp u = a[j], v = a[j + (m >> 1)] * tmp[j - i];
					a[j] = u + v, a[j + (m >> 1)] = u - v;
				}
		}
		if (op == -1) rep(i, 0, n) a[i] = cp{ a[i].x / n,a[i].y / n };
	}
	vector<V> multiply(vector<V> A, vector<V> B, int op = 0) {
		if (op) reverse(all(A));
		int lena = A.size(), lenb = B.size(), len = 1;
		while (len < lena + lenb) len <<= 1;
		aa = vector<cp>(len), bb = vector<cp>(len);
		rep(i, 0, lena) aa[i] = cp{ (double)A[i],0 };
		rep(i, 0, lenb) bb[i] = cp{ (double)B[i],0 };
		FFT(aa, len, 1), FFT(bb, len, 1);
		rep(i, 0, len) aa[i] = aa[i] * bb[i];
		FFT(aa, len, -1); A.clear();
		if (!op) rep(i, 0, len) A.pb((ll)(aa[i].x + 0.5)); else
			rep(i, lena - 1, lena + lenb - 2 + 1) A.pb((ll)(aa[i].x + 0.5));
		return A;
	}
};
FT<ll> fft;
\end{lstlisting}
\subsection{NTT}
\begin{lstlisting}[language=C++]
const int M = 1 << 17 << 1;
int a[M], b[M];

struct NTT{
	static const int G = 3, P = 1004535809; //P = C*2^k + 1
	int N, na, nb, w[2][M], rev[M];
	ll kpow(ll a, int b){
		ll c = 1;
		for (; b; b >>= 1,a = a * a % P) if (b & 1) c = c * a %P;
		return c;
	}
	void FFT(int *a, int f){
		rep(i, 0, N) if (i < rev[i]) swap(a[i], a[rev[i]]);
		for (int i = 1; i < N; i <<= 1)
			for (int j = 0, t = N / (i << 1); j < N; j += i << 1)
				for (int k = 0, l = 0, x, y; k < i; k++, l += t)
					x = (ll) w[f][l] * a[j+k+i] % P, y = a[j+k], a[j+k] = (y+x) % P, a[j+k+i] = (y-x+P) % P;
		if (f) for (int i = 0, x = kpow(N, P-2); i < N; i++) a[i] = (ll)a[i] * x % P;
	}
	void work(){
		int d = __builtin_ctz(N);
		w[0][0] = w[1][0] = 1;
		for (int i = 1, x = kpow(G, (P-1) / N), y = kpow(x, P-2); i < N; i++) {
			rev[i] = (rev[i>>1] >> 1) | ((i&1) << (d-1));
			w[0][i] = (ll)x * w[0][i-1] % P, w[1][i] = (ll) y * w[1][i-1] % P;
		}
	}
	void doit(int *a, int *b, int na, int nb){ // [0, na)
		for (N = 1; N < na + nb - 1; N <<= 1);
		rep(i, na, N) a[i] = 0;
		rep(i, nb, N) b[i] = 0;
		work(), FFT(a,0), FFT(b,0);
		rep(i, 0, N) a[i] = (ll)a[i] * b[i] % P;
		FFT(a, 1);
		//rep(i, 0, N) cout << a[i] << endl;
	}
} ntt;
\end{lstlisting}
\subsection{欧拉函数}
\begin{lstlisting}[language=C++]
/*
ŷ����ʽEuler(n)=n/(1-1/p1)(1-1/p2)(��)
��[1,n-1]����n���ʵ����ĸ��� 
*/ 

//�󵥸�ŷ������ֵ 
int euler(int n){
	int ans = 1,i;
	for (i = 2; i * i <= n; i++){
		if (n % i == 0){
			n /= i;
			ans *= i - 1;
			while (n % i == 0){
				n /= i;
				ans *= i;
			}
		}
	}
	if (n > 1) ans *= n - 1;
	return ans;
}
//ɸ��������1..nŷ������ֵ
\end{lstlisting}
\subsection{线性筛素数}
\begin{lstlisting}[language=C++]
void sift_prime(bool notprime[],int N){
	vector<int> prime; 
	memset(notprime,false,sizeof(bool)*N);
	notprime[0] = notprime[1] = 1;
	for (int i = 2; i < N; ++i){
		if (!notprime[i]) prime.push_back(i);
		for (int j = 0; i * prime[j] <= N && j < prime.size(); ++j){
			notprime[i * prime[j]] = 1;
			if (i % prime[j] == 0) break; //speed up linear time
		}
	}
}
\end{lstlisting}
\subsection{高斯消元}
\begin{lstlisting}[language=C++]
int guass(int n){
	int ans=1,t,tmp;
	for (int i=0; i<n; i++){
		for (int j=i+1; j<n; j++){
			while (mat[j][i]){
				t=mat[i][i]/mat[j][i];
				for (int k=0; k<n; k++){
					tmp=mat[i][k];
					tmp-=t*mat[j][k];
					mat[i][k]=tmp;
				}
				for (int k=0; k<n; k++) swap(mat[i][k],mat[j][k]);
				ans=-ans;
			}
		}
		if (mat[i][i]==0) return 0;	
		ans*=mat[i][i];
	}
	return ans;
}
\end{lstlisting}
